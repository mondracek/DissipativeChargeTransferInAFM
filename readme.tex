\documentclass[a4paper,10pt]{report}
%\usepackage[utf8]{inputenc}

%% Title Page
\title{Simulating dynamics of a bi-stable charge state interacting with an oscillating tip
for the purpose of a paper on `Quantum dissipation driven by electron transfer within a single molecule investigated by means of atomic force microscopy'}
\author{Martin Ondr\'{a}\v{c}ek, Pavel Mal\'{y}, Tom\'{a}\v{s} Man\v{c}al, Pavel Jel\'{\i}nek}

%\addtolength{\textwidth}{\marginparsep}
\addtolength{\textwidth}{\marginparwidth}
%\setlength{\marginparsep}{0pt}
\setlength{\marginparwidth}{0pt}
\setlength{\oddsidemargin}{0pt}
\setlength{\evensidemargin}{0pt}

\begin{document}
\maketitle
\setlength{\parindent}{0pt}
\setlength{\leftskip}{0pt}
\setlength{\rightskip}{0pt}
\setlength{\parskip}{0.5\baselineskip}
\setlength{\oddsidemargin}{0pt}
\setlength{\evensidemargin}{0pt}
%\begin{abstract}
%Code to generate simulated data for the paper `Quantum dissipation driven by electron transfer within a single molecule investigated by means of atomic force microscopy
%\end{abstract}

File {\verb AFM_charge-hopping_wavedata.py } contains the code that was used to carry out the simulations shown in our paper (Berger et al., Nature Chemistry 2019).
More specifically, it generates data shown in Figure 4 of the paper. The simulated spectra we show in Figure 3 can be also obtained with the code after editing a single line in it, namely removing or commenting out the line that sets {\verb zpoints=1 }.

The code requires {\it Python3} and the {\it NumPy} library to run. We specifically used it with Python 3.6.2 and Numpy 1.16.1, but we expect it to run with most other versions of the software too, as it uses only pretty standard tools of the language and the library. Generating some plots on the fly also requires {\it Matplotlib}, but the corresponding lines are now commented out in the code, as we used the code to generate ASCII files with numerical data and then created the plots using external tools.

The code has no user interface. Any input parameters have to be changed by directly editing the source code. The same holds for any adjustments as to what the code outputs and in what format.

The algorithm implements the model described in the Supplementary Online material of our paper, in particular what we call ``the full version'' of the model.

\section*{Output}
The output files generated by the code are in general named according to the template {\it prefix}\verb _ {\it parameters}\verb .dat , where the ``prefix'' indicates what kind of data the file contains and the part denoted as parameters encodes numerical values of some of the parameters of the model. The names of the parameters used to compose the names of the output files may unfortunately differ from both the names of the corresponding variables inside the code and the names of the quantities we discuss in the paper. Sorry for that, we left the code as we used it ourselves, without any attempt to make it user friendly. The correspondence between the variables in the code and quantities in the paper is indicated in the list below, under section ``Physical parameters''.

\subsection*{Output files}
\begin{flushleft}
\newlength{\leftskipbackup}\newlength{\parindentbackup}
\setlength{\leftskipbackup}{\leftskip}\setlength{\parindentbackup}{\parindent}
{\verb de_*.dat } $\Delta E(t) = E_A(t) - E_B(t) + E_0$\\
\setlength{\leftskip}{4ex}\setlength{\parindent}{-4ex}
{\verb V1_*.dat } $q, V_A(q)$ (potential well on site A)\\
{\verb V2_*.dat } $q, V_B(q)$ (potential well on site B)\\
{\verb wave1_*t?.?????.dat } $q, \rho_A(q)$ probability densities on site A for time $t=$?.?????~$\mu$s\\
{\verb wave2_*t?.?????.dat } $q, \rho_B(q)$ probability densities on site B for time $t=$?.?????~$\mu$s\\
{\verb rate1_*.dat } $t, k_{AB}(t)$, where  $k_{AB} (t) = \int k_{AB} (q,t) \rho_B (q) dq$ is the $q$-integrated immediate rate for the A$\leftarrow$B transfer\\
{\verb rate2_*.dat } $t, k_{BA}(t)$, where  $k_{BA} (t) = \int k_{BA} (q,t) \rho_A (q) dq$ is the $q$-integrated immediate rate for the B$\leftarrow$A transfer\\
{\verb occ1_*.dat } $t, p_A(t)$ (occupation probabilities on site A)\\
{\verb occ2_*.dat } $t, p_B(t)$ (occupation probabilities on site B)
\end{flushleft}
%\setlength{\leftskip}{\leftskipbackup}\setlength{\parindent}{\parindentbackup}

After removing the line with {\tt zsize=1}, files with spectra will be generated. These files are named {\verb z-df-exc_*.dat } and contain $z_0, \Delta f(z_0), E_\mathrm{diss}(z_0)$, for $z_0$ starting with {\verb leftz } and increasing by a step of {\verb dz }.


\section*{Physical parameters}
The list below indicates the correspondence between internal variables of the code and the notation that was used for the parameters of the model in the paper.
\begin{flushleft}
{\verb kBT } = $k_\mathrm{B} T$ (but see the note about a loop over temperatures below)\\
\setlength{\leftskip}{4ex}\setlength{\parindent}{-4ex}
{\verb reorgE } = $\Lambda$\\%, reorganization energy for the fast (Marcus) degrees of freedom
{\verb hwrel } = 2$E_\Lambda$\\%, reorganization energy for the slow (explicit) degrees of freedom
{\verb rate } = $k_0$ %, Marcus transfer rate
(but rescaled to temperature so that the numerical value is true for $T = 1.2$~K while it corresponds to the same transfer matrix element $|M|$ for other temperatures\\
{\verb Lambda } = $K_\Lambda$ \\%, time constant for relaxation of the explicit degree of freedom (wave packets in the parabolic potential wells)
{\verb alpha } = $Q_\mathrm{tip}$%, point charge on the tip
\end{flushleft}
%\setlength{\leftskip}{\leftskipbackup}\setlength{\parindent}{\parindentbackup}

\section*{`Technical' parameters (some of them)}

\subsection*{height ($z_0$-)grid}
\begin{flushleft}\verb  \\\vspace{-\baselineskip}
\setlength{\leftskip}{4ex}\setlength{\parindent}{-4ex}
{\verb zsize } = number of points in the $\Delta f(z)$ and $E_\mathrm{diss}$(z) spectra. When set to one, no spectra are generated and instead the temporal evolution of quantities like $\rho_{A,B}(q,t)$, $p_{A,B}(t)$, etc. is written for a tip oscillating around the position given by {\verb z0 }.\\
{\verb leftz } = starting value of {\verb z0 }---the mean height around which the the tip oscillates---for calculating the $\Delta f(z_0)$ and $E_\mathrm{diss}(z_0)$ spectra\\
{\verb dz } = incremental step of the tip height in $\Delta f(z_0)$ and $E_\mathrm{diss}(z_0)$ spectra
\end{flushleft}
%\setlength{\leftskip}{\leftskipbackup}\setlength{\parindent}{\parindentbackup}

\subsection*{temporal ($t$-)grid}
\begin{flushleft}
{\verb dt } = time step\\
\setlength{\leftskip}{4ex}\setlength{\parindent}{-4ex}
{\verb tsteps } = number of time steps in the simulation (so {\verb tsize*tsteps } is the total simulation time)

In general, time dependent quantities are not written out for the whole simulation but rather only for some number of steps at the end of the simulation. This is controlled by parameter {\verb tsize }.

{\verb tsize } = number of time steps at the end of the simulation to be written to the output.
\end{flushleft}
%\setlength{\leftskip}{\leftskipbackup}\setlength{\parindent}{\parindentbackup}

\subsection*{explicit DOF ($q$-)grid}
\begin{flushleft}
{\verb qleft } = leftmost (smallest) value of the $q$-coordinate
\setlength{\leftskip}{4ex}\setlength{\parindent}{-4ex}
{\verb dq } = increment of the $q$-coordinate grid
{\verb qsize } = size (number of points) of the $q$-coordinate grid
\end{flushleft}
%\setlength{\leftskip}{\leftskipbackup}\setlength{\parindent}{\parindentbackup}

The code contains a {\verb for } loop that allows the user to generate results for several different values of some parameter (or more parameters, if one uses a loop over ordered tuples). Currently, it is used for temperature, so the {\verb kBT } variable is set inside the loop and the default value set previously in the code is being ignored.

Units used in the code: $10^{-22}$~J for energy (but $E_\mathrm{diss}$ and $V_{A,B}$ already outputs in meV!), pN for forces, \AA{} for length, elementary charge for electric charge, $\mu$s for time.

\end{document}          
